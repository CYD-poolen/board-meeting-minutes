% -*- compile-command: "pdflatex cyd-2015-02-26.tex"; -*-

\documentclass[a4paper,12pt]{article}
\usepackage[utf8]{inputenc}
\usepackage[swedish]{babel}
\usepackage{enumerate}
\usepackage{hyperref}
\usepackage{pdfpages}

\begin{document}
\section{CYD-poolens styrelsemöte 2015-09-22}

\def\arraystretch{1.3}
\begin{tabular*}{\textwidth}{@{\extracolsep{\fill} }l c r}
Närvarande & Ämbete & Funktion \\
\hline\\[-0.4cm]
Kristina Arkad & Teknisk chef ISY & Ledamot\\
Christian Luckey & Systemadministratör & Justerare\\
Hans-Filip Elo & Systemadministratör & Sekreterare\\
Erik Sköld & Webbmaster, Y-sektionen & Justerare\\
Alf Segersäll & Chef LiU-IT kvarter Valla nord & Ordförande\\
Torun Berlind & Utbildningsledare EF-nämnden & Ledamot\\
Michael Sörsäter & Infochef, D-sektionen & Ledamot\\
Oskar Sundström & Ledamot D-webb, D-sektionen & Ledamot\\
Victor Karlsson Sehlin & Webbmaster, D-sektionen & Ledamot\\[2cm]
\end{tabular*}


\begin{enumerate}
\item Mötets öppnande.
\item Eventuella adjungeringar till mötet.
\item Alf Segersäll valdes till ordförande.
\item Hans-Filip Elo valdes till sekreterare samt Christian Luckey och Erik Sköld till justerare.
\item Dagordningen godkändes.
\item Föregående protokoll.
  \begin{itemize}
  \item Påtryckningar har gjorts på Akademiska hus angående luftkvalitén men de har ej åtgärdat något sedan förra mötet. CYD-poolen har en handlingsplan för att bättre kunna ställa krav på universitetet och Akademiska hus.
  \item Stadgar bordlagda till detta möte från föregående.
  \item CYD-poolen samt D- och Lingsektionerna har nu fått igång sina nya webbplatser.
  \end{itemize}
\item Verksamhetsberättelse.
  \begin{itemize}
    \item Ny webbplats cyd.liu.se.
    \item Lingsektionen har ny webbplats.
    \item D-sektionen har ny webbplats.
    \item Strul under sommaren. CYD-poolen har upplevt strömavbrott i hallen där hårdvara finns. Detta har lett till att vissa maskiner, fysiska som virtuella, krånglat. Ett stort arbete har gjorts för att återställa allt - och CYD-poolen är nästan i mål med det.
    \item Ett till masterprogram har access till CYD-poolen.
    \item Förbättrad SSL/TLS-säkerhet på HTTPS-sidor i CYD-poolen.
    \item Automatiska penetrationstester på alla Wordpress-webbplatser.
    \item Arbetat och arbetar kontinuerligt med Obelix för att uppdatera deras webbplats till modernare versioner av ramverket Django. Detta för att kunna uppgradera webbserver till Debian 8 Jessie.
  \end{itemize}
\item Detta ligger i CYDs pipeline just nu:
  \begin{itemize}
    \item Mailserver åt Linkdagarna strax klar - väntar på LiU-IT.
    \item Mer redundant switching.
    \item HTTPS för alla webbplatser i CYD.
    \item Fortsatt arbete med Linux på desktop.
    \item CYD håller på och rekryterar en ny person som systemtekniker - vilket är klart i början av oktober.
  \end{itemize}

\item Luftkvalitén i CYD.
~\\
Påtryckningar till Akademiska hus som rengjort AC-anläggningar och återkom sedan efter någon månad - Akademiska hus säger att lokalerna uppfyller kraven på ventliation sett till antalet tillåtna personer i lokalen. Vi har beställt koldioxidmätare och kommer föra statistik på luftkvalitén i CYD. Erik föreslår forserad ventilation på luncherna. Alf tar med sig det till Akademiska hus och hör huruvida det är genomförbart. Christian och Hans-Filip kommer sätta upp stationen som samlar in data.

\item CYD-poolens stadgar.
~\\
Stavfel fanns i stadgarna - korrigeras. Stadgarna godkännes med förbehåll att de stavfel som fanns korrigeras. Korrigerade stadgar bifogas tillsammans med detta protokoll.

\item Övriga frågor.
  \begin{itemize}
    \item Erik intresserad av den mailserver som sätts upp för Linkdagarna. Detta tas efter mötet med systemteknikerna för CYD.
    \item Oskar frågar kring Let's encrypt\footnote{\url{https://letsencrypt.org/}} och när CYD kan tänkas ha HTTPS på alla webbplatser. Let's encrypts tidplan är att ha allmän tillgänglighet i slutet av november.
    \item Torun har frågor kring finansieringen av CYD. Vem som ska står för den utöver delfinansieringen från programnämnderna? Svaret är att det inte är klart än och man inväntar svar från nya "Datorsalsprojektet" som drivs av Martin Eneling.
  \end{itemize}
\item Mötet avslutas.
\end{enumerate}

\vspace{2cm}
Justeras:
~\\
~\\
~\\
~\\
~\\
\noindent\begin{tabular}{ll}
\makebox[0.5\textwidth]{\hrulefill} & \makebox[0.5\textwidth]{\hrulefill}\\
Erik Sköld & Datum\\[1.5cm]
\makebox[0.5\textwidth]{\hrulefill} & \makebox[0.5\textwidth]{\hrulefill}\\
Christian Luckey & Datum\\
\end{tabular}

% ------
\newpage
\section{Bilagor}

På nästa sida följer CYD-poolens stadgar fastslagna vid möte 2015-09-22.

\includepdf[pages={-}]{stadgar.pdf}

\end{document}
