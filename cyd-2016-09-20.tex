% -*- compile-command: "pdflatex cyd-2016-02-15.tex"; -*-

\documentclass[a4paper,12pt]{article}
\usepackage[utf8]{inputenc}
\usepackage[swedish]{babel}
\usepackage{enumerate}
\usepackage{hyperref}
\usepackage{pdfpages}
\usepackage{graphicx}

\begin{document}
\section{CYD-poolens styrelsemöte 2016-09-20}

\def\arraystretch{1.3}
\begin{tabular*}{\textwidth}{@{\extracolsep{\fill} }l c r}
Närvarande & Ämbete & Funktion \\
\hline\\[-0.4cm]
Kristina Arkad & Teknisk chef ISY & Ledamot\\
Christian Luckey & Systemadministratör & Justerare\\
Hans-Filip Elo & Systemadministratör & Ordförande\\
Martin Estgren & Systemadministratör & Sekreterare\\
Tea Nygren & Utbildningsledare DM-nämnden & Justerare\\
Jesper Wrang & Webmaster - D-sektionen & Ledarmot\\ 
Mikael Ångman & Sekreterare/Infochef - D-Sektionen & Adjungant\\
Patrik Sletmo & D-LAN - D-sektionen & Adjungant\\
Nathalie Svensson & Info/Webbansvarig - Lingsektionen & Ledarmot\\
Sven Engström & Webmaster - Y-Sektionen & Ledarmot\\

\end{tabular*}

\begin{enumerate}
\item Mötets öppnande.
\item Eventuella adjungeringar till mötet.
\item Hans-Filip valdes till ordförande.
\item Martin Estgren valdes till sekreterare.
\item Christian Luckey och Tea Nygren valdes till justerare.
\item Genomgång av föregående mötesprotocoll.
\item Verksamhetsberättelse.
	\begin{itemize}
	\item Redundanta switchar har installerats.
	\item Primär mailserver har uppdaterat med TLS och bättre anti-email spoofing.
	\item En ny mailserver redo att ta in studentorganisationer med externa domäner har satts upp.
	\item Klimatdatan har presenterats och diskuterats med arbetsmiljöombud.
	\item CYD-poolen har tagit in Y6 som ny studentorganisation och hostar deras hemmisda.
	\item Teknisk hjälp när Lingsektionens hemsida gick ner.
	\item Samtliga wordpress hemsidor som hostas har fått HTTPS via Let's Encrypt.
	\item Thinlink har installerats på datorerna i datorsalen.
	\end{itemize}
\item Detta ligger i CYDs pipeline just nu:
	\begin{itemize}
	\item Rekrytering av en ny systemadministratör till CYD-poolen.
	\item Uppgradering av IT-infrastrukturen (virtualiseringslösning 3.0).
	\item Flytta routing av nät.
	\end{itemize}
\item "Datorsalsprojektet" och CYD-poolens framtid (det vi vet)
	\begin{itemize}
 	\item Studentorganisationerna förhåller sig positiva till den service och tjänster som CYD-poolen tillhandahåller i dagsläget.
	\item CYD-administratörerna ska komminucera med de andra studentdatorsalanrna och sammanställa en lista på de tjänster som riskerar att försvinna efter årskiftet. 
	\item Möjligheten att driva CYD-poolen som en digital verkstad.
		\begin{itemize}
		\item CYD-administratörerna ålägger sig att skapa ett dokument där studentorganisaionterna får möjlighet att utrycka vad de tycker är bra med CYD-poolen i dagsläget. Detta dokument skall sedan skickas till LIU-IT, LinTek och Nämnderna.
		\item Tea tar frågan gällande digtal verkstad till nämnderna.
		\end{itemize}
	\end{itemzie}
\item Återkoppling angående luftkvalitén.
	\begin{itemize}
	\item Christian visar grafer över hur koldioxidhalten i CYD-poolen under en typisk dag.
	\item Christian ålägger sig att prata med Universitetsförvaltningen gällande utställning av mikrovågsvagn för att lätta på mikrotrycket under lunchtiderna.
	\end{itemize}
\item Återkoppling angående SLA (Service Level Agreement).
	\begin{itemize}
	\item Styrelsen går med på utformingen av CYD-poolens SLA.
	\item CYD-administratörerna ålägger sig att skicka ut SLA till de berörda studentorganisationerna.
	\end{itemize}
\item Övriga frågor.
\item Nästa Sammanträde sätts preliminärt till 2017-02-08
\item Mötet avslutas.
\end{enumerate}

\vspace{2cm}
\noindent
Justeras:
~\\
~\\
\noindent\begin{tabular}{ll}
\makebox[0.5\textwidth]{\hrulefill} & \makebox[0.5\textwidth]{\hrulefill}\\
Namn & Datum\\[1.5cm]
\makebox[0.5\textwidth]{\hrulefill} & \makebox[0.5\textwidth]{\hrulefill}\\
Namn & Datum\\
\end{tabular}
% ------
\end{document}
