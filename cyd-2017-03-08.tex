% -*- compile-command: "pdflatex cyd-2017-03-08.tex"; -*-

\documentclass[a4paper,12pt]{article}
\usepackage[utf8]{inputenc}
\usepackage[swedish]{babel}
\usepackage{enumerate}
\usepackage{hyperref}
\usepackage{pdfpages}
\usepackage{graphicx}

\begin{document}
\section{CYD-poolens styrelsemöte 2017-03-08}

\def\arraystretch{1.3}
\begin{tabular*}{\textwidth}{@{\extracolsep{\fill} }l c r}
Närvarande & Ämbete & Funktion \\
\hline\\[-0.2cm]
Christian Luckey & Systemadministratör & Ordförande\\
Martin Estgren & Systemadministratör & Sekreterare\\
Jepser Wrang & Webbmaster, D-sektionen & Ledamot/Justerare \\
Nathalie Svensson & Info/Webbansvarig, Lingsektionen & Ledamot/Justerare \\
Martin Enegling & Chef Digitala Resursenheten & Ledamot \\
Sven Engström & Webbmaster, Y-sektionen & Ledamot\\
Tea Nygren & Utbildningsledare DM-nämnden & Ledamot\\
Torun Berlind & Utbildningsledare EF-nämnden & Ledamot\\
Patrik Sletmo & D-LAN, D-sektionen & Adjugant \\

\end{tabular*}
\begin{enumerate}
\item Mötets öppnande.
\item Eventuella adjungeringar till mötet.
\item Christian Luckey väljs till ordförande.
\item Martin Estgren väljs till sekreterare.
\item Jesper Wrang och Nathalie Svensson väljs till justerare.
\item Genomgång av föregående mötesprotokoll.
\item Verksamhetsberättelse.
	\begin{itemize}
	\item Flyttad routing av nät.
	\item Ny brandvägg.
	\item Servermigration, Dessa pekar fortfarande på CYD-poolen.
		\begin{itemize}
		\item Y6
		\item Lingsektionen
		\item LiU Formula Student
		\item TackLING
		\item Obelix
		\item Y-fadderiet
		\item Pi-Balen
		\end{itemize}
	\item Pratat framtid med Joakim Nejdeby, Martin Enegling och LinTek.
		\begin{itemize}
		\item Lobbat för att datorsalen inte ska bli bokningsbar.
		\end{itemize}
	\item Ny Windows-image i salen.
	\item Hans-Filip har slutat.
	\end{itemize}
\item Detta ligger i CYDs pipeline just nu:
	\begin{itemize}
	\item Rekrytering av en ny systemadministratör till CYD-poolen.
	\item Övergång till Linux i salen.
	\end{itemize}
\item Datorsalsprojektet och CYD-poolens framtid.
	\begin{itemize}
		\item Martin beskriver bakgrunden till Datorsalsprojektet samt vad det innebär för framtiden.
		\begin{itemize}
		\item CYD-poolen blir en datorsal för självstudier. I undantagsfall skall det gå att boka lokalen för undervisning. Detta sker genom kontakt med projektförvaltaren (Martin Eneling) som skickar förfrågan vidare till Joakim Nejdeby. 
		\item CYD-administratörernas nya roll under Digitala resursenheten (hädanefter benämnt "DRS") blir att jobba som IT-tekniker i samtliga datorsalar under DRS:s administration på Campus Valla.
		\item Datorsalar kan bli "plussade" av institutionerna om specialmjukvara eller hårdvara behövs i utbildningssyfte. I gengäld får studenter med relevanta kursregistreringar inom den aktuella institutionen förtur till bokning av datorer i den "plussade" salen.
		\item Köksdelen av CYD-poolen blir kvar tills vidare men blir inte ett ansvarsområde för DRS.
		\end{itemize}
	\end{itemize}
\item Status gällande servermigration från CYD-poolen.
	\begin{itemize}
	\item Lingsektionen är i slutskedet med sin migration till Lysator. Hemsidan fungerar men pekar fortfarande på CYD-poolen då lite arbete kvarstår.
	\item Fem sidor har flyttats för D-sektionens räkning. Inloggning för sektionssidor är tillgängligt men med en improviserad lösning.
	\item Y-sektionens sektionssida kör redan hos Lysator. Y-fadderiet och Y6 är fortfarande kvar hos CYD-poolen. Migration kommer ske för båda inom snar framtid.
	\item Y6 ansvarar för Pi-Balen.
	\item TackLING skall kontaktas av CYD-administratörerna gällande status av servermigration.
	\item LiU Formula Student är aktivt igång med sin servermigration.
	\item Obelix får dispens av LiU-IT till sommaren.
	\item Framtida drift och underhåll av studentorganisationernas IT-miljöer har fortfarande inte en konkret lösning.
		\begin{itemize}
		\item Martin Enegling undersöker ifall serverhosting -- eventuellt -- kan erbjudas av LiU-IT i form utav en virtuell server.
		\item Nämnderna lyfter frågan kring framtiden för hosting och drift av studentorganisationerna IT-miljöer.
		\end{itemize}
	\end{itemize}
\item Övriga frågor.
	\begin{itemize}
		\item CYD-administratörerna skickar med att sektionerna bör äska för framtida behov av hosting och underhåll av IT-miljöerna till utbildningsnämnderna.
		\item CYD-poolens styrelse upplöses. Framtida beslut och önskemål går via Objektrådet med representant från LinTek, eller via medarbetare inom DRS.
	\end{itemize}
\item Mötet avslutas.
\end{enumerate}

\vspace{2cm}
\noindent
Justeras:
~\\
~\\
\noindent\begin{tabular}{ll}
\makebox[0.5\textwidth]{\hrulefill} & \makebox[0.5\textwidth]{\hrulefill}\\
Namn & Datum\\[1.5cm]
\makebox[0.5\textwidth]{\hrulefill} & \makebox[0.5\textwidth]{\hrulefill}\\
Namn & Datum\\
\end{tabular}
% ------
\end{document}
